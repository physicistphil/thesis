\graphicspath{{Chapters/Chapter_intro/}}

\chapter{Introduction}
\label{ch:intro}

Nuclear fusion is potentially a fantastic long-term energy source. On Earth, the fuel is abundant, the waste manageable, and is otherwise clean. The most easily accessible fusion reaction for us is the fusion of two heavy hydrogen isotopes: deuterium and tritium, with one and two neutron respectively. The reaction produces a neutron with 14.1 MeV of energy and a helium-4 nucleus with 3.5 MeV. In terms of mass of the reactants, the energy density of fusion is roughly seven orders of magnitudes higher than hydrocarbon fuels. Reaching the conditions for this fusion reaction requires of the plasma sufficiently high density $n$, temperature $T$, and confinement time $\tau$. Since the inception of the controlled fusion program, progress has been made towards improving these parameters \cite{wurzel_progress_2022}, but much work is left to be done to put the power of the sun in the palm of our hands.

\section{Nuclear fusion via magnetic confinement}

The current dominant structure for fusion plasmas is the donut-shaped tokamak \cite{john_wesson_tokamaks_2004}. This device operates by first imposing a toroidal field with external magnets, like bending a solenoid into a circle. Toroidal devices make intuitive sense: magnetic field can only confine plasmas perpendicular to the field via the Lorentz force $F = q(E + v \times B)$, so the third axis is wrapped around to itself so that the magnetic field lines are closed. A current is induced in this toroidal direction to create an orthogonal poloidal field component to cancel out particle drifts. As of this writing, tokamaks are in the lead for plasma performance in magnetic confinement fusion \cite{wurzel_progress_2022}.

Tokamaks, however, have two major drawbacks: the large toroidal current and the complicated geometry. At minimum, a breakeven reactor requires many millions of Amperes of circulating current \cite{creely_overview_2020}. This current is a source of free energy that, in situations called disruptions, can be dumped into an electron beam that causes catastrophic damage to the wall as well as structural damage via induced currents. These disruptions are thought by some to be a showstopper for the tokamak program (personally, I am confident disruptions will be figured out given enough time and money), but can be avoided by instead building a toroidal device that has a (mostly) externally set field. Such a device is called a stellarator.

Devices that are difficult and expensive to build will not be economically competitive -- costs are very important to the viability of fusion reactors \cite{schwartz_value_2023}. Geometrically simple, linear devices, like mirror machines, may be the path forward despite intrinsically worse confinement. 

The magnetic mirror approach to fusion \cite{Post_1987} operates on the principle of conservation of the magnetic moment $\mu = \frac{W_\perp}{B}$. When a particle enters a region of higher field, $W_\perp$ increases by conservation of $\mu$. Given sufficiently high $W_\perp$ relative to $W_\parallel$, by conservation of energy ($W_\perp + W_\parallel = \text{constant}$) $W_\parallel$ must decrease and, on occasion, the particle must stop and reverse direction. Thus, some particle trapping can be achieved by creating a solenoid with two high field magnets at either end. Some particles can escape, and if the loss rate is rapid enough, deplete a portion of the velocity space leading to a ``loss cone'' distribution which can then drive many vicious instabilities. The primary focus of magnetic mirror research has been on plugging this hole in the distribution function and mitigating the interchange instability.

\section{Confinement degradation from instabilities and turbulence}

All fusion plasmas are hot relative to the outside world and are situated within a vacuum, which implies the existence of temperature and density (and thus pressure) gradients. These gradients provide free energy to drive instabilities, which in some cases cascade to smaller length scales and dissipate. This process is turbulence, and it occurs inside every fusion-relevant plasma that we know of. These turbulent eddies can be large and are the fundamental limit on cross-field transport for fusion reactors. Effort must be made to study and characterize this turbulence, and if possible, suppress it. 

\section{Accelerating research using machine learning}

Machine learning, though a nascent field for quite some time, has exploded in popularity since the advent of deep neural networks trained on GPUs \cite{krizhevsky_imagenet_2017}. Machine learning is effectively curve fitting, though, as compute is becoming cheaper and GPUs more powerful, larger models can be trained with greater capabilities. Fundamentally, machine learning, and in particular deep learning, scale very well with data and dimensionality. Plasma devices collect a significant amount of data and so do the simulations used to interpret the experimental data. These observations raise the following questions (which have been bothering me since before my PhD): 
\begin{enumerate}
	\item Is there additional information in our diagnostics and simulations that we cannot exploit using conventional means?
	\item Can we guide a scientific campaign (or fusion program) using this additional information without human intervention?
	\item What are the fundamental limitations on the pace of fusion reactor development anyways?
\end{enumerate}

I attempt to partially answer questions 1 and 2 in this thesis. As for question 3, I suspect that the fundamental limitations are raw manpower and the difficulty of building complicated machines. Machine learning and focusing on mirror machines may help address these fundamental limitations.

%The study of plasmas is characterized by temperamental experiments, incomplete and untrustworthy diagnostics, and difficult theory. Machine learning provides a new tool to tackle these challenges. Currently, many scientific fields are learning how to use machine learning techniques effectively and how they can be used to accelerate scientific progress. 

\section{Motivation for this work}

\section{Outline of the dissertation}
Chapter \ref{ch:background} describes the Large Plasma Device, the configuration of the machine, and the diagnostics used in this thesis. It also provides background on some relevant instabilities and a very brief introduction to neural networks. 

Chapter \ref{ch:mirror_turbulence} details a study of mirror turbulence and transport using the older barium oxide cathode. Multiple mirror ratios and lengths were analyzed, with the conclusion being that cross-field particle flux reduced at higher mirror ratios, and no evidence of the interchange instability was seen.

Chapter \ref{ch:ml-dataruns} describes the process of collecting data from randomly-set LAPD configurations, the diagnostics collected, and the biases inherent in the data. The dataset collected here is used in the two subsequent chapters. The chapter also covers the data cleaning process for some of the diagnostics. 

Chapter \ref{ch:isat-predict} is a thorough machine learning study on predicting ion saturation current -- $I_\text{sat}$ -- in LAPD mirrors using neural networks. This is possibly the first time a global optimization like this has been performed in magnetized plasma device. In addition, trends were inferred by scanning along various combinations of machine inputs. The important features of this work are the validation of model predictions and thorough uncertainty quantification. Fundamentally, this is a demonstration that machine learning can be used to extract insights from data in a complicated, multivariate physical system. 

Chapter \ref{ch:ebm} takes a different angle on machine learning than what was done in chapter \ref{ch:isat-predict}. Instead of predicting a single output given machine inputs, we instead learn a probability distribution over the data using an energy-based model. Using this model, we can reconstruct any arbitrary combination of diagnostics, machine settings, or anything in the input space -- including hallucinating discharges altogether. The flexibility of the learned energy function is demonstrated, and this flexibility may be a way to combine theory and experiment to improve predictions. 

Appendix \ref{app:0d-mirror} describes a small 0d reactor optimization project in collaboration with Kunal Sanwalka, based on a spreadsheet by Cary Forest. This project was undertaken to learn how to use Jax and SymPy, and also gain some intuition on mirror-based fusion reactors.
