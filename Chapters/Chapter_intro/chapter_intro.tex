\graphicspath{{Chapters/Chapter_intro/}}

\chapter{Introduction}
\label{ch:intro}

\section{Nuclear fusion: brief introduction}

\subsection{Mirror machines as a faster way to fusion power}

%Given the LAPD parameters in this study (tables ref), the collision frequencies are sufficiently high such that the mirror is in the gas-dynamic regime: losses out of the mirror throat are governed by collisions. The mirror length must exceed the scattering into the loss cone\cite{Ivanov_2013}:
%\begin{equation}
%    L > \lambda_{ii} \cdot \ln{R} / R
%\end{equation}
%These collisions populate the loss cone and maintain a Maxwellian distribution, eliminating the possibility of loss-cone-driven instabilities like the AIC or DCLC instabilities that have been observed in hotter devices
%
%\section{Instabilities in mirrors}
%\subsection{Interchange}
%
%At low $\beta$, as we are in the LAPD, the flute instability is a purely electrostatic perturbation. The growth rate of interchange is approximately \cite{Ryutov_2011}
%\begin{equation}
%    \Gamma_0 = \frac{c_s}{\sqrt{L_M L}}
%\end{equation}
%where $L_M$ is the length of the mirror section measured where the curvature flattens out on either end of the cell and $L$ is the total length of the plasma. The distinction between the length of the mirror and the length of the plasma is important because the instability is driven by the pressure of the plasma where the magnetic field has curvature, but the inertia of the plasma comes from it's total volume. Just for some numbers, assuming $c_s \approx 10$ km/s, $L_M \approx 7$ m, and $L \approx 17$ m, one expects a growth rate of $\Gamma_0 \approx 1.2$ kHz.
%
%The interchange instability occurs when the density gradient and magnetic curvature, when integrated along the field line, point in the same direction. 
%Stabilization by finite Larmor radius (FLR) effects \cite{Post_1987}:
%\begin{equation}
%    \frac{a_i}{r_p} > \frac{m^{1/2}}{m-1} \frac{r_p}{L}
%\end{equation}
%Our ions are so cold that FLR effects only become significant when $m \gtrsim 300$
%
%Line-tying and vortex stabilization 
%
%\subsection{Alfvén ion cyclotron (AIC)}
%\subsection{Drift cyclotron loss cone (DCLC)}



\section{The Large Plasma Device at UCLA}

\subsection{Diagnostics at the LAPD}

\subsubsection{Langmuir probes: $I_\text{sat}$, sweeps, triple probes}

\subsubsection{The 288 GHz heterodyne interferometer}

\subsubsection{Thomson scattering}

\subsubsection{Fast framing camera}

\subsection{Data acquisition}

\section{Analysis techniques}

\subsection{Spectral analysis}

\subsection{Correlation techniques}

\subsection{Machine learning}

\subsubsection{Neural networks}
