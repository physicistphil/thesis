\chapter{Conclusions}
\label{ch:conclusion}

\section{Mirror turbulence and transport}

Several studies were undertaken, exploring turbulence and transport in mirror configurations, optimization and trend inference in mirror configurations using machine learning, and generative modeling of machine state and diagnostics using energy-based models. 

The primary goal of the study of turbulence in transport was to examine the coupling of the curvature-induced interchange instability with drift waves (and other modes) in the LAPD. The interchange instability was not observed: all modes seen were also present in the flat $M=1$ case, ruling out curvature drive. The instability was likely not visible because of the many stabilization mechanisms present, such as FLR stabilization for azimuthal mode numbers $m>4$, line-tying to the cathode which reduces the growth rate, a stabilization mechanism caused by ionization of neutral gas, and shear flow which disrupts the mode structure. Although no evidence for interchange mode was seen, though other interesting behavior was present. 

Surprisingly, the cross-field transport as measured by the $\tilde{E} \times B$ particle flux decreased as the mirror ratio was increased from $M=1$ to $M=2.68$. This decrease in particle flux was largely caused by a decrease in power of the density fluctuations for mirror ratios up to $M=1.9$. The decrease in particle flux and fluctuation power is, in-part, likely caused by the increasing density gradient scale length $L_n$ when the mirror ratio is increased (the plasma expands radially). 

Coincident peaks in density and magnetic fluctuations were observed at 12 kHz and up, increasing with mirror ratio. These fluctuations were identified as drift-Alfv\'en waves, with the frequency set by the maximum Alfv\'en half-wavelength with the magnetic field averaged over the length of the plasma. Additional modes were seen at 3 kHz in density and 6 kHz in magnetics. The origin of these fluctuations could not be determined, but could potentially be a combination drift waves, a rotational interchange instability, a nonlinear instability, or a conducting wall mode. 

This work indicates that the edge of a mirror fusion reactor may be fairly stable, particularly if it has larger gradient scale length.

\section{Machine learning on the LAPD}

Two machine learning studies were performed on the LAPD: predicting a time-averaged ion saturation current ($I_\text{sat}$ signal) and generative modeling (creating a joint distribution) of time-series diagnostics and machine state. Both of these studies utilized a diverse dataset collected on the LAPD. Machine configurations were sampled -- using Latin hypercube sampling (LHS) -- by varying gas puff settings, magnetic field configurations, and discharge parameters. This dataset is the first time on the LAPD, and possibly the first time in magnetized plasmas, where machine settings have been set using a sampling technique instead of a conventional grid search. This diversity in machine configurations lead to a large diversity in plasma discharge behavior. Although only 44 machine configurations were sampled out of 67 dataruns in total, this dataset was sufficient to train models to predict broader LAPD behavior.

By training a relatively simple neural network (NN) model, machine settings were mapped to $I_\text{sat}$ signals averaged from 10 to 20 ms. This simple model was able to learn trends over the data despite the limited diversity of the dataset. Trends inferred by varying discharge voltage or magnetic field configuration agreed with intuition and expectations. When validated on other LAPD data, the model predictions matched reality fairly well. By performing a comprehensive search over machine inputs, any function of $I_\text{sat}$ can be optimized. This model was used to find optima in axial variation as measured by the standard deviation of $I_\text{sat}$ signals. The predicted machine configurations for minimal, maximal, and intermediate axial variation were applied to the LAPD. The predicted $I_\text{sat}$, although fairly inaccurate on an absolute scale, nevertheless predicted the trends in axial variation very well. This model could be used to find ideal conditions for experimental studies requiring a particular $I_\text{sat}$ profile. 

This simple model used a $\beta$-NLL loss and ensembles to measure uncertainty. The uncertainty was not calibrated on an absolute scale, but is useful when comparing two different predictions of the model: greater model uncertainty corresponded to increased error of the prediction. The thorough uncertainty quantification used in this study, and the distinction between aleatoric and epistemic uncertainty, is unique for machine learning studies in plasma physics. 

An energy-based model (EBM) was also trained on this diverse dataset. EBMs are an uncommon generative model and this work is the first time EBMs has been applied to plasmas. Instead of predicting a single average $I_\text{sat}$ value, additional time-series signals were included. These time-series signals included discharge current and voltages, five diodes, the interferometer, and $I_\text{sat}$. The EBM learned an energy surface that represented the joint distribution of all these signals along with machine configuration. This multimodal energy surface was parameterized by a hybrid fusion architecture, separately learning input representations and combining them later, with some early crossover for probe positions. This energy surface was sampled using Langevin dynamics. Generating conditional samples by freezing gradients was demonstrated. A novel method of conditional sampling via energy surface modification was also demonstrated, leading to higher quality, physically realistic, samples. Unconditional samples indicated that the EBM learned all modes of the data distribution, but struggled to learn the mass associated with each mode.

In terms of scope, the flexibility of EBMs makes their analysis comparable to traditional data analysis. A couple ways of using these EBMs were demonstrated. Interferometer signals were reconstructed, both with and without additional time-series signals. Including other diagnostics, such as diodes, increased the accuracy of the predicted time traces. This use of diode signals is notable because they are uncalibrated and not ananlyzed, yet the ML model was able to exploit this information that would otherwise require extensive manual analysis. Given that the EBM learned a joint probability distribution, any diagnostic signal can be reconstructed using any combination of machine settings or other diagnostics signals. This flexibility is unparalleled in machine learning in plasma physics. In addition to conditional sampling, the energy surface was directly examined. By scanning the $I_\text{sat}$ probe x position for a given data point, several dips in the energy surface were occasionally seen, indicating the x-coordinate symmetry intrinsic in the plasma. The inclusion of single-dimensional parameters in the joint distribution allows this novel analysis of the energy function. Typical EBMs only have one type of input, such as an image, which intrinsically has a high-dimensional surface. This energy surface scan highlights the ability of the EBM to learn symmetries and the ability for it to learn multivalued functions.

%This dissertation detailed several projects spanning turbulence and transport in mirror machines to machine learning applications for data analysis. The study of turbulence and transport did not yield the coupling between drift waves and the interchange mode as desired. However, this lack of evidence for the interchange mode highlights how complicated a system LAPD mirrors are, and how many different stabilization mechanisms can come into play. The reduced particle flux with increased mirror fields was also unexpected given the unfavorable curvature of mirror configurations.

\section{Future work}

The interchange instability may be visible in the LAPD by increasing the plasma $\beta$ and magnetic curvature. Increased $\beta$ would increase the growth rate and also increase the curvature by opposing the applied field, and a larger difference in mirror and midplane fields could further encourage growth of the interchange mode. Once this mode is excited, a study can be performed to analyze the coupling of interchange to other modes. Achieving a $\beta$ of 1\% -- an order of magnitude higher than present in this study -- is possible in the LAPD using the new LaB$_6$ cathode. 

Disentangling the different modes present in LAPD mirrors proved to be quite difficult. Further experiments with additional probes are needed to accurately measure parallel wavelength of different modes; the work presented here used only two probes which gave an effective average over all modes present. In particular, differentiating between flute like $k_\parallel=0$ and finite-$k$ modes would prove beneficial in mode identification. Additional diagnostics and probes outside of the mirror cell would also be useful in determining if modes exist that are confined to the mirror cell. 

Although different length mirrors were briefly examined, the dependence of the modes present on mirror length can be more thoroughly studied. This length dependence (and thus average field) could prove beneficial when attempting to differentiation the instabilities present. 

Additional diagnostics and a diversity of machine conditions are helpful, but simulations would be the most useful in understanding which modes are possible or present. The computational approach could range from an eigenvalue solver to gyrokinetics, the latter of which would be capable of modeling the turbulence present in the device (nearly all fluctuation power was far below the cyclotron frequencies so the gyrokinetic approach is appropriate). As performed in previous work \cite{Friedman_2013}, a two-fluid edge transport code (BOUT++ \cite{dudson_bout_2009}) could also be used as an intermediate approach. 

If this mirror turbulence project were to be taken up again on the current LaB$_6$ cathode, machine learning could have a very prominent role to play in finding the stability boundary. Like the work done on optimizing $I_\text{sat}$ profiles, the optimization function could instead be strength of an instability instead of axial variation. The current dataset collected -- although diverse -- likely does not contain sufficient information to find the stability boundary over the multidimensional LAPD actuator or settings space. More data would need to be taken for this approach to work.

The machine learning work performed here could be substantially improved by collecting a greater diversity of LAPD configurations. This greater diversity could be enabled by automated adjustment of LAPD settings (given some safe operating range). Automated setting changes and randomized probe positions could enable much faster data acquisition and sampling of LAPD parameter space since grid-based scans are no longer necessary with an ML approach. This increased diversity would dramatically improve ML model performance. The cathode and thus plasma conditions also change over hours, days, and weeks -- inclusion of these effects would improve the generalizability of learned plasma behavior. A dataset of 29 million discharges spanning four years has already been collected and may be able to be used to categorize and account for intrinsic changes in LAPD plasmas.

Additional diagnostics, such as Langmuir sweeps, triple probes, floating potential, Thomson scattering, or the diamagnetic loop, could be included as well. The inclusion of additional diagnostics enables diagnostic calibration after data collection. For example, $I_\text{sat}$ and triple probe $T_e$ measurements could be sampled along an interferometer chord to determine the effective area of the Langmuir probe, assuming the Langmuir probes are relatively calibrated to each other. Probe $T_e$ measurements could be calibrated by sampling at the Thomson scattering measurement volume. 

The flexibility of EBMs allows joint sampling with other distributions. As long as there is some overlap with experimental data, EBMs trained on simulation data can be sampled to generate discharges that are both experimentally plausible and within the simulated distribution. This joint sampling would allow interrogation of plasma physics only available to simulations (such as transport coefficients or the entire multi-species plasma distribution function) but grounded in experimental measurements. I see this combination as potent and a promising next step for deepening our understanding of experimental plasmas.

\section{Accelerating fusion science}

The machine learning work in this paper partially addressed the questions laid out in the introduction, in short: how can we accelerate fusion science? Evidently, an optimization can be performed by collecting randomized data, to a degree of qualitative success. Despite the poor quantitative performance, it is often good enough to know which direction to move in, as many steps in the right direction can put you on the desired target. In that way machine learning could help accelerate progress -- I hope this work encourages other plasma and fusion groups to consider exploring beyond conventional experimental campaigns and increase the diversity of experiments. The ability to explore and infer trends using ML is also very useful -- this trend discovery is often the goal of many experimental campaigns in fusion and plasma physics. 

The energy-based modeling work also indicates another path to accelerating fusion science. The learned energy function was evidently useful and flexible for reconstructing diagnostics (or theoretically performing any inverse problem) and extracting insight. This model, as demonstrated for conditional sampling, can be combined with any other energy-based model. This model could be trained on another experimental dataset or simulation results, and the joint distribution sampled from. The performance improvement by including uncalibrated and otherwise unused diagnostics while sampling hints that much more information is available to be autonomously exploited from our devices, even if this information is not directly interpretable. If an ML-first approach were used to study fusion plasmas, dramatic gains could be achieved in optimization and understanding. 

In conclusion, when done properly, machine learning can be a very useful tool for extracting insight from data and has the potential to dramatically accelerate fusion science. If fusion science is accelerated, then our iteration on experiments can likewise be faster, and I believe mirror machines are most amenable to this rapid iteration.



























