%%%%%%%%%%%%%%%%%%%%%%%%%%%%%%%%%%%%%%%%%%%%%%%%%%%%%%%%%%%%%%%%%%%%%%%%
%                                                                      %
%                          PRELIMINARY PAGES       %
%                                                                      %
%%%%%%%%%%%%%%%%%%%%%%%%%%%%%%%%%%%%%%%%%%%%%%%%%%%%%%%%%%%%%%%%%%%%%%%%

\title{Study and optimization of turbulence and transport in mirror configurations in the Large Plasma Device}

\author         {Philip Nathanael Travis}
\department     {Physics and Astronomy}
\degreeyear		{2025}
\nocopyright

%%%%%%%%%%%%%%%%%%%%%%%%%%%%%%%%%%%%%%%%%%%%%%%%%%%%%%%%%%%%%%%%%%%%%%%%

\chair          {Troy\ Carter}
\member         {Jacob\ Bortnik}
\member         {Christoph\ Niemann}
\member         {Paulo\ Alves}


%%%%%%%%%%%%%%%%%%%%%%%%%%%%%%%%%%%%%%%%%%%%%%%%%%%%%%%%%%%%%%%%%%%%%%%%
\dedication{To humanity\\and Altair, my dog}
\acknowledgments {I acknowledge peeps}

%%%%%%%%%%%%%%%%%%%%%%%%%%%%%%%%%%%%%%%%%%%%%%%%%%%%%%%%%%%%%%%%%%%%%%%

\vitaitem	{2017}{B.S. (Engineering Physics), University of Illinois at Urbana-Champaign}
\vitaitem	{2018}{Masters (Physics), University of California, Los Angeles}
\vitaitem	{2025}{Ph.D. (Plasma Physics), University of California, Los Angeles}

%%%%%%%%%%%%%%%%%%%%%%%%%%%%%%%%%%%%%%%%%%%%%%%%%%%%%%%%%%%%%%%%%%%%%%%%

\publication{Pubs}


%%%%%%%%%%%%%%%%%%%%%%%%%%%%%%%%%%%%%%%%%%%%%%%%%%%%%%%%%%%%%%%%%%%%%%%%


\abstract{

The primary goal of this thesis is to work towards accelerating and automating fusion science. The combination of the physical flexibility of mirror machines with the optimization and information-exploitation potential of machine learning may be a potent one and is explored here. Progress towards this overarching goal is accomplished by studying turbulence in a mirror machine, optimizing mirror configurations using machine learning, and developing a highly extendable and flexible model for correlating and interpreting diagnostic signals. 

In the Large Plasma Device (LAPD) at UCLA, a study of turbulence and transport in mirror configurations was undertaken. Using the flexible nature of the LAPD field configuration, several different mirror ratios from $M=1$ to $M=2.68$ were studied. Langmuir and magnetic probes were used to measure profiles of density, temperature, potential, and magnetic field. Particle flux measurements were also taken. The goal of this work was to see the interaction of interchange modes with drift waves, but no such interchange modes were observed likely because of the many stabilization phenomena present. This fact, along with reduced cross-field particle flux, indicate  that a sufficiently cold edge of a simple mirror may have less cross-field transport than one would expect.

For the purposes of machine learning, a partially-randomized dataset was collect in the LAPD mirror configurations. The goal was to maximize the diversity of data to cover the largest portion of machine operation space as possible. Using this collected dataset, neural network (NN) ensembles with uncertainty quantification were trained to predict time-averaged ion saturation current ($I_\text{sat}$ — proportional to density and the square root of electron temperature) at any position within the dataset domain. This model was then used to optimize the device for strong, intermediate, and weak axial variation of $I_\text{sat}$. In addition, this model was used to infer trends in the effect on $I_\text{sat}$ of LAPD controls. This model and optimization were validated on followup experiments, yielding qualitative and, at times, quantitative, agreement. This investigation demonstrated that, using ML techniques, insights can be extracted from experiments and magnetized plasmas can be globally optimized. The primary goals of this work were to provide an example of a solid, validated machine learning study and demonstrate how ML can be useful in understanding operating plasma devices.

Using this same randomized dataset, a generative model was trained to learn a probability distribution. In particular, energy-based models (EBMs) provide a powerful and flexible way of learning relationships in data by constructing an energy surface. In this work, a CNN- and attention-based multimodal EBM was trained on time series and single-dimensional data. This EBM learned all distributional modes of the data but with some differences in probability mass. Via conditional sampling of the model using a novel, auxiliary energy function technique, diagnostic reconstruction is demonstrated. In addition, the inclusion of additional diagnostics improved reconstruction error and generation quality, showing that even uncalibrated, unanalyzed diagnostics can provide useful information. Fundamentally, this work demonstrated the flexibility and efficacy of EBM-based generative modeling of laboratory plasma data, and demonstrated practical use of EBMs in the physical sciences.
}

%%%%%%%%%%%%%%%%%%%%%%%%%%%%%%%%%%%%%%%%%%%%%%%%%%%%%%%%%%%%%%%%%%%%%%%%
