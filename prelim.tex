%%%%%%%%%%%%%%%%%%%%%%%%%%%%%%%%%%%%%%%%%%%%%%%%%%%%%%%%%%%%%%%%%%%%%%%%
%                                                                      %
%                          PRELIMINARY PAGES       %
%                                                                      %
%%%%%%%%%%%%%%%%%%%%%%%%%%%%%%%%%%%%%%%%%%%%%%%%%%%%%%%%%%%%%%%%%%%%%%%%

\title{Study of turbulence and transport in and optimization of \\mirror configurations in the Large Plasma Device}

\author         {Philip Travis}
\department     {Physics}
\degreeyear		{2025}
%\nocopyright

%%%%%%%%%%%%%%%%%%%%%%%%%%%%%%%%%%%%%%%%%%%%%%%%%%%%%%%%%%%%%%%%%%%%%%%%

\chair          {Troy A. Carter}
\member         {Jacob Bortnik}
\member         {Christoph Niemann}
\member         {Eduardo Paulo Jorge da Costa Alves}


%%%%%%%%%%%%%%%%%%%%%%%%%%%%%%%%%%%%%%%%%%%%%%%%%%%%%%%%%%%%%%%%%%%%%%%%
\dedication{\emph{To humanity\\and Altair, my dog}}
\acknowledgments {I am fortunate to have many to thank for making my PhD such an enjoyable and satisfying part of my life. Shout out to my homies Gurleen Bal, Kamil Sklodowski, and Yhoshua Wug for providing much-needed support and camaraderie over all these years -- graduate school would have been very different (maybe painful!) without you guys.

I would also like to extend gratitude to all those in the nfmfmhdut community and the SULI mafia for their support and friendship, particularly Becky Masline and Shawn Zamperini for building a community for us over covid. In no particular order: Kunal Sanwalka, Sam Frank, Nick McGreivy, Ben Israeli, Garrett Prechel, Andrew Maris, Allen Wang, Kenny Gage, Ryan Chaban, Mason Yu, Tyler Cote, Genevieve DeGrandchamp, Alex Leviness, Nirbhav Chopra (UIUC gang), Oak Nelson. If I forgot anyone, I apologize, and be sure grab me next APS for a drink.

Thank you to my labmates, classmates, and collaborators for fostering a warm community: Jeff Robertson, Mel Abler, Tom Look, Javi Serrano, Jesus Perez, Josh Larson, Henry Wong, Erik Everson, Tim Vanhoomissen, Ryan Buckley, Henry Wong, Sarah Chase, Tri Luong, Lukas Rovige, Yuchen Qian, Bobby Dorst, Jess Pilgram, Jia Han, Seth Dorfman, Angelica Ottaviano, Patrick Crandall, Gary Wan, Jess Gonzalez, Graeme Sabiston, Mary Konopliv, Ani Thuppul, and Adam Collins. And in particular, shoutout to Gio Rossi for helping pass the comprehensive exam with a hilariously imbalanced competition and Kyle Callahan and Tom Neiser for being such great roommates at conferences. To those in the astronomy department, thank you for accepting a wayward physicist as one of your own (at least on the camping trips). Thanks to the lab staff for their willingness to help running experiments and answering late-night texts: Shreekrishna Tripathi, Steve Vincena, Pat Pribyl, Zoltan Lucky, Marvin Drandell, and Tai Sy Ly. On behalf of all of us, I'd like to thank lab mom Meg Murphy for the administrative and emotional support (and food!) over all these years -- it's hard to imagine what the lab would be like without her.

Friends and family, thank you for providing support and homes to crash at when I needed to: Lydia and Andy Wilson (and Savannah and Shepard), Tim and Allison Travis, Sean Lyon, Claire Baum, Erik Wessel, Rory Bentley (spacebros), Ada Morral, Antonett Prado, and most of all Piper (Bernese) and Blake (terrier mix). And anti-collaborator Altair (Golden-Bernese mix), thank you for keeping me sane in all the important ways and driving me nuts in all the others.

Thank you professors Walter Gekelman, George Morales, and Derek Schaeffer for your instruction and insight.
To committee members, Jacob Bortnik and Paulo Alves, thank you for your insight and support in my exploration of machine learning techniques. And thank you to Chris Niemann for answering my late-night texts when troubleshooting the Thomson scattering system.
Lastly, and most importantly, I'd like to thank my advisor, Troy Carter, for supporting me despite my wanderings in ML territory. I am certain that the work I have done here would not have been possible anywhere else under any other advisor, and for that I am immeasurably grateful. 

Oh, and thank you U.S. taxpayers: this work was performed at the Basic Plasma Science Facility, which is a DOE Office of Science, FES collaborative user facility and is funded by DOE (DE-FC02-07ER54918) and the National Science Foundation (NSF-PHY 1036140).

}

%%%%%%%%%%%%%%%%%%%%%%%%%%%%%%%%%%%%%%%%%%%%%%%%%%%%%%%%%%%%%%%%%%%%%%%

\vitaitem	{2017}{B.S. in Engineering Physics, University of Illinois at Urbana-Champaign}
\vitaitem	{2018}{M.S. in Physics, University of California, Los Angeles}
\vitaitem	{2017-2025}{Graduate student researcher, University of California, Los Angeles}

%%%%%%%%%%%%%%%%%%%%%%%%%%%%%%%%%%%%%%%%%%%%%%%%%%%%%%%%%%%%%%%%%%%%%%%%

\publication{P. Travis, T. Carter, \emph{Turbulence and transport in mirror geometries in the Large Plasma Device}, Journal of Plasma Physics. 91(1):E40 (2025). doi:10.1017/S0022377825000029}

\publication{Y. Qian, W. Gekelman, P. Pribyl, T. Sketchley, S. Tripathi, Z. Lucky, M. Drandell, S. Vincena, T. Look, P. Travis, T. Carter, G. Wan, M. Cattelan, G. Sabiston, A. Ottaviano, R. Wirz, \emph{Design of the Lanthanum hexaboride based plasma source for the large plasma device at UCLA}, Rev. Sci. Instrum. 94, 085104 (2023). doi:10.1063/5.0152216}

\presentation{P. Travis, J. Bortnik, T. Carter, \emph{An open dataset from the Large Plasma Device for machine learning and profile prediction}, poster presentation, 66$^\text{th}$ Annual Meeting of the APS Division of Plasma Physics, Atlanta, GA (2024)}

\presentation{P. Travis, T. Look, L. Rovige, C. Niemann, P. Pribyl, T. Carter, \emph{Predicting profiles in LAPD mirror configurations}, poster presentation, 65$^\text{th}$ Annual Meeting of the APS Division of Plasma Physics, Denver, CO (2023)}

\presentation{P. Travis, S. Vincena, P. Pribyl, T. Carter, \emph{Developing a generative ML model for LAPD trend inference and profile prediction}, 64$^\text{th}$ Annual Meeting of the APS Division of Plasma Physics, Spokane, WA (2022)}

\presentation{P. Travis, T. Carter, \emph{Upgrading LAPD diagnostic pipelines for training generative ML models}, poster presentation, 63$^\text{rd}$ Annual Meeting of the APS Division of Plasma Physics, Pittsburg, PA (2021)}

\presentation{P. Travis, T. Carter, \emph{Automated Langmuir sweep analysis using machine learning}, contributed talk, 62$^\text{nd}$ Annual Meeting of the APS Division of Plasma Physics, online, FL (2020)}

\presentation{P. Travis, T. Carter, \emph{Study of turbulence and transport in magnetic mirror geometries in the LAPD}, poster presentation, Transport Task Force, San Diego, CA (2018)}


%%%%%%%%%%%%%%%%%%%%%%%%%%%%%%%%%%%%%%%%%%%%%%%%%%%%%%%%%%%%%%%%%%%%%%%%


\abstract{

The primary goal of this thesis is to work towards accelerating and automating fusion science. The combination of the physical flexibility of mirror machines with the optimization and information-exploitation potential of machine learning may be a potent one and is explored here. Progress towards this overarching goal is accomplished by studying turbulence in a mirror machine, optimizing mirror configurations using machine learning, and developing a highly extendable and flexible model for correlating and interpreting diagnostic signals. 

In the Large Plasma Device (LAPD) at UCLA, a study of turbulence and transport in mirror configurations was undertaken. Using the flexible nature of the LAPD field configuration, several different mirror ratios from $M=1$ to $M=2.68$ were studied. Langmuir and magnetic probes were used to measure profiles of density, temperature, potential, and magnetic field. Particle flux measurements were also taken. The goal of this work was to see the interaction of interchange modes with drift waves, but no such interchange modes were observed likely because of the many stabilization phenomena present. This fact, along with reduced cross-field particle flux, indicate  that a sufficiently cold edge of a simple mirror may have less cross-field transport than one would expect.

For the purposes of machine learning, a partially-randomized dataset was collect in the LAPD mirror configurations. The goal was to maximize the diversity of data to cover the largest portion of machine operation space as possible. Using this collected dataset, neural network (NN) ensembles with uncertainty quantification were trained to predict time-averaged ion saturation current ($I_\text{sat}$ — proportional to density and the square root of electron temperature) at any position within the dataset domain. This model was then used to optimize the device for strong, intermediate, and weak axial variation of $I_\text{sat}$. In addition, this model was used to infer trends in the effect on $I_\text{sat}$ of LAPD controls. This model and optimization were validated on followup experiments, yielding qualitative and, at times, quantitative, agreement. This investigation demonstrated that, using ML techniques, insights can be extracted from experiments and magnetized plasmas can be globally optimized. The primary goals of this work were to provide an example of a solid, validated machine learning study and demonstrate how ML can be useful in understanding operating plasma devices.

Using this same randomized dataset, a generative model was trained to learn a probability distribution. In particular, energy-based models (EBMs) provide a powerful and flexible way of learning relationships in data by constructing an energy surface. In this work, a CNN- and attention-based multimodal EBM was trained on time series and single-dimensional data. This EBM learned all distributional modes of the data but with some differences in probability mass. Via conditional sampling of the model using a novel, auxiliary energy function technique, diagnostic reconstruction is demonstrated. In addition, the inclusion of additional diagnostics improved reconstruction error and generation quality, showing that even uncalibrated, unanalyzed diagnostics can provide useful information. Fundamentally, this work demonstrated the flexibility and efficacy of EBM-based generative modeling of laboratory plasma data, and demonstrated practical use of EBMs in the physical sciences.
}

%%%%%%%%%%%%%%%%%%%%%%%%%%%%%%%%%%%%%%%%%%%%%%%%%%%%%%%%%%%%%%%%%%%%%%%%
